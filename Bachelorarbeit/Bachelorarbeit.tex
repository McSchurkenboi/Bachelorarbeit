\documentclass[12pt]{report}
\usepackage[german]{babel}
\usepackage[utf8]{inputenc}
\usepackage{fontspec}
\usepackage{fancyhdr}
\usepackage{mathrsfs}
\usepackage{amssymb}
\usepackage{amsmath}
\usepackage{amsfonts}
\usepackage{caption}
\usepackage{tikz}
\usepackage{tikz-uml}
\usetikzlibrary{automata,arrows,positioning,shapes}
\tikzstyle{activity} = [rectangle, draw, text centered, text width=7em, rounded corners, minimum height=2em]
\tikzstyle{dia} = [diamond, draw]
\tikzstyle{invis} = []
\usepackage{textcomp}
\usepackage{verbatim}
\usepackage{booktabs, tabularx}
\usepackage{graphicx}
\usepackage{multicol}
\usepackage{paralist}
\usepackage{enumitem}
\renewcommand\tabularxcolumn[1]{m{#1}}
\setlist[itemize,1]{label=$\bullet$}
\setlist[itemize,2]{label=$\bullet$}

\pagenumbering{arabic}
\pagestyle{fancy}
\rhead{Analyse von Projektlastenheften}
\renewcommand{\footrulewidth}{1pt}
\renewcommand{\arraystretch}{0.6}

\begin{document}
\begin{titlepage}
\raggedright
\begin{large}
Entwurf und Implementierung einer Werkzeugunterstützung zur sprachlichen Analyse und automatisierten Transformation von Projektlastenheften im Kontext der Automobilindustrie
\end{large}

\vfill\vfill\vfill\vfill
An der Fachhochschule Dortmund\\
\vfill
im Fachbereich Informatik\\
\vfill
Studiengang Informatik\\
\vfill
Vertiefung Praktische Informatik\\
\vfill
erstellte Thesis\\
\vfill\vfill\vfill\vfill
zur Erlangung des akademischen Grades\\
\vfill
Bachelor of Science\\
\vfill
B. Sc.\\
\vfill\vfill\vfill\vfill
von Aaron Schul, \\
\vfill
geboren am 24.06.1997\\
\vfill
und Felix Ritter\\
\vfill
geboren am 31.08.1997\\
\vfill\vfill
Betreuung durch:\\
\vfill
Prof. Dr. Sebastian Bab\\
\vfill
Dortmund, 28.02.2019\\
\end{titlepage}

\newpage
\begin{abstract}
demotext
\end{abstract}

\renewcommand{\abstractname}{Abstract}
\begin{abstract}
demotext
\end{abstract}

\newpage
\tableofcontents
\newpage
\listoftables
\listoffigures
\newpage

\begin{center}
\textbf{Danksagung} \\
Ich danke meiner Mama und seiner Mama, das was er gesagt hat.
\end{center}
\newpage

\newpage
\chapter{Einführung}
\section{Motivation}
Im betrieblichen Umfeld liegen zu Beginn eines jeden Entwicklungsprojektes für neue Produkte die Aufgaben und Ziele für die Entwicklung als Dokumente vor. Forschungsergebnisse finden Anwendung in der Vorentwicklungsphase, in der die Eignung der Erkenntnisse für neue Produkte eines Unternehmens evaluiert wird. Die Produktentwicklung unterliegt dabei bestimmten Kriterien und Faktoren, die den unternehmerischen Erfolg beeinflussen. Neben betriebswirtschaftlichen Einflüssen wie der Einordnung des Produktes in der Wertschöpfungskette sind es dabei besonders technische Anforderungen an das Produkt, die definiert und während der Produktentwicklung eingehalten werden müssen. Verschiedenste Akteure aus einem Umternehmen sind dabei an der Festlegung der Anforderungen an ein Entwicklungsprojekt bzw. Produkt beteiligt. 

In der Automobilindustrie betrifft dieser Ablauf zumeist die Entwicklung neuer Fahrzeugkomponenten, heutzutage meist elektronische und mechanische Bausteine. Diese Bausteine werden dabei nicht sämtlich vom Fahrzeughersteller (OEM) selbst, sondern durch eine Vielzahl von Zulieferern produziert und entwickelt. Die Produktspezifikationen liegen meist digital als Texte, Tabellen und Grafiken vor und werden an den Zulieferer übermittelt.
Nach dem Entwicklungsprozess steht dann die (Serien-)entwicklung und -fertigung des Produktes für das Ausrollen in großen Stückzahlen an den Hersteller, der das zugelieferte Produkt dann in seinen Produkten verwendet. Um dies zu erreichen, müssen während des gesamten Prozesses die Anforderungen, die das Systemumfeld des  Fahrzeugherstellers hat, berücksichtigt und eingehalten werden.

Die Anforderungen an das Produkt, etwa technische Rahmenbedingungen, werden dabei von vielen verschiedenen Domänenexperten beim OEM formuliert und in das sogenannte Pflichtenheft für die Entwicklung eingetragen. Beteiligte sind etwa Produktdesigner, Ingenieure und Systemtechniker, die an verschiedenen Stellen im Lastenheft Anforderungen an eine Komponente festlegen. Diese Beteiligten sind in der Regel auf ihren Bereich spezialisiert und nicht interdisziplinär, zudem gibt es sprachliche Eigenheiten der Autoren und unternehmensinterne Richtlinien für die Formulierung, die das Verständnis erschweren können. Demzufolge sammeln sich im Lastenheft verschiedenste Merkmale einer Komponente, die aber nicht im Bezug zueinander stehen und sich im schlimmsten Fall gegenseitig ausschließen. 

Durch diese fachliche Breite und Tiefe der Spezifikationen im Pflichtenheft, aber auch durch den Umfang des Lastenheftes von mehreren tausend Seiten, kommt es häufig insbesondere zu Verständnisproblemen auf Seite des Zulieferers. Die Gewichtung einzelner Anforderungen in einem größeren Systemkontext fällt dort schwer, da nun Projektteammitglieder, die an der Entstehung des Lastenheftes nicht beteiligt waren, dieses verstehen und ein Produkt entwickeln sollen, dass möglichst alle Anforderungen berücksichtigt. In Texten muss also nach Zusammenhängen und Bezügen zwischen mehreren Anforderungen gesucht werden, damit die Korrektheit des späteren Produktes gewährleistet ist.

Die Analyse von Zusammenhängen zwischen Anforderungen stellt dabei aus Gründen der Effizienz ein Problem dar, wenn jeder Beteiligte von Hand die für ihn relevanten Anforderungen aus dem Lastenheft extrahieren muss. Auch müssen die Lastenhefte an die Formulierungen und Ausdrucksweisen für Requirements-Management im Unternehmen angepasst werden. Bislang gibt es jedoch kaum Werkzeugunterstützung, die effiziente Möglichkeiten zur automatisierten Überarbeitung und Anpassung einzelner Anforderungen aus dem Dokument bietet. Ansätze aus dem \textit{Natural-Language-Proessing} (NLP) stellen gleichzeitig vielversprechende Forschungsfelder in der Informatik dar, die eine solche automatisierte Verarbeitung auf Basis von Sprachanalyse ermöglichen. Syntax und Semantik der einzelnen Sätze und Zusammenhänge in Texten können auf Basis aktueller Trends wie Machine-Learning und dynamischer Programmierung zunehmend besser abgebildet werden.

\section{Hypothese}
Hypothese dieser Arbeit ist, dass sich mithilfe von NLP Lastenhefte effizeint automatisiert verarbeiten lassen, womit die Arbeit von Requirements-Engineers, aber auch von Beteiligten an der Entwicklung beim Verständnis der Anforderungen erleichtert wird. Insbesondere die Auswertung der Syntax ist dabei für ein tieferes Verständnis von Textzusammenhängen, also von verschiedenen Anforderungen, relevant.

\section{Methodik}
\section{Inhalt}
\section{Autoren}
\chapter{Stand der Technik}
\section{Rückblick auf die Projektarbeit, Grundlagen zu NLP und Ontologien}
\section{Verwandte Arbeiten}
\chapter{Betriebliches Umfeld - Hella Use-Case}
\section{Re-Prozesse bei Hella und allgemein in Firmen}
\section{Betriebliche Anforderungen}
\section{Ansatz und Konzept unserer Werkzeuge}
\chapter{R2B-Converter}
\section{Architektur Klassen und Verteilung der Ressourcen}
\section{Implementierung (bisschen Code, GUI, Listenarchitektur, Workflow für User}
\section{mögliche Erweiterungen}
\section{Test}
\subsection{Methodik}
\subsection{Durchführung}
\subsection{Ergebnisse}
\chapter{Delta-Analyse}
\section{Architektur}
\section{Implementierung}
\section{mögliche Erweiterungen}
\section{Test}
\subsection{Methodik}
\subsection{Durchführung}
\subsection{Ergebnisse}
\chapter{Evaluation}
\section{Auswertung der Testresultate}
\section{Ziel erreicht? Hypothese reviewen und schwafeln}
\section{Mehrwert?}
\chapter{Fazit}
\section{Zusammenfassung}
\section{Ausblick}


\newpage
\begin{thebibliography}{20}

\end{thebibliography}
\end{document}
