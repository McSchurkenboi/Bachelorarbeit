\documentclass[12pt]{report}
\usepackage[german]{babel}
\usepackage[utf8]{inputenc}
\usepackage{fontspec}
\usepackage{fancyhdr}
\usepackage{mathrsfs}
\usepackage{amssymb}
\usepackage{amsmath}
\usepackage{amsfonts}
\usepackage{caption}
\usepackage{tikz}
\usepackage{tikz-uml}
\usetikzlibrary{automata,arrows,positioning,shapes}
\tikzstyle{activity} = [rectangle, draw, text centered, text width=7em, rounded corners, minimum height=2em]
\tikzstyle{dia} = [diamond, draw]
\tikzstyle{invis} = []
\usepackage{textcomp}
\usepackage{verbatim}
\usepackage{booktabs, tabularx}
\usepackage{graphicx}
\usepackage{multicol}
\usepackage{paralist}
\usepackage{enumitem}
\renewcommand\tabularxcolumn[1]{m{#1}}
\setlist[itemize,1]{label=$\bullet$}
\setlist[itemize,2]{label=$\bullet$}

\pagenumbering{arabic}
\pagestyle{fancy}
\rhead{Analyse von Projektlastenheften}
\renewcommand{\footrulewidth}{1pt}
\renewcommand{\arraystretch}{0.6}

\begin{document}
\begin{titlepage}
\raggedright
\begin{large}
Entwurf und Implementierung einer Werkzeugunterstützung zur sprachlichen Analyse und automatisierten Transformation von Projektlastenheften im Kontext der Automobilindustrie
\end{large}

\vfill\vfill\vfill\vfill
An der Fachhochschule Dortmund\\
\vfill
im Fachbereich Informatik\\
\vfill
Studiengang Informatik\\
\vfill
Vertiefung Praktische Informatik\\
\vfill
erstellte Thesis\\
\vfill\vfill\vfill\vfill
zur Erlangung des akademischen Grades\\
\vfill
Bachelor of Science\\
\vfill
B. Sc.\\
\vfill\vfill\vfill\vfill
von Aaron Schul, \\
\vfill
geboren am 24.06.1997\\
\vfill
und Felix Ritter\\
\vfill
geboren am 31.08.1997\\
\vfill\vfill
Betreuung durch:\\
\vfill
Prof. Dr. Sebastian Bab\\
\vfill
Dortmund, 28.02.2019\\
\end{titlepage}

\newpage
\begin{abstract}
demotext
\end{abstract}

\renewcommand{\abstractname}{Abstract}
\begin{abstract}
demotext
\end{abstract}

\newpage
\tableofcontents
\newpage
\listoftables
\listoffigures
\newpage

\begin{center}
\textbf{Danksagung} \\
Ich danke meiner Mama und seiner Mama, das was er gesagt hat.
\end{center}
\newpage

\newpage
\chapter{Einführung}
\section{Motivation}
\section{Hypothese}
\section{Methodik}
\section{Inhalt}
\section{Autoren}
\chapter{Stand der Technik}
\section{Betriebliches Szenario}
\section{Verwandte Arbeiten}
\chapter{Kontext, Use-Case und Aufgabenstellung unserer Entwicklung}
\chapter{Konzepte R2BC und DA}
\chapter{R2B-Converter}
\section{Anforderungen}
\section{Implementierung}
\section{mögliche Erweiterungen}
\section{Test}
\chapter{Delta-Analyse}
\section{Anforderungen}
\section{Implementierung}
\section{mögliche Erweiterungen}
\section{Test}
\chapter{Fazit}
\section{Zusammenfassung}
\section{Ausblick}


\newpage
\begin{thebibliography}{20}

\end{thebibliography}
\end{document}
